\section{Introduction}
\label{introduction}

Nearly 537 million adults worldwide are diagnosed with diabetes mellitus with a consistently rising prevalence \cite{federation2021idf}. Type 2 Diabetes (T2D) accounts for the clear majority with 90\% of all diabetes cases \cite{federation2021idf}. For T2D patients, medication, activity, and dietary habits are key elements in the management of keeping blood glucose levels within a regulated range to avoid adverse events. However, unregulated insulin treatment and unhealthy lifestyle habits may increase the risk of severe hypoglycemia leading to serious short-term consequences \cite{evans2013health}. In particular, nocturnal hypoglycemia (NH) is a common complication, with over 50\% of severe hypoglycaemic episodes occurring during night-time \cite{graveling2017risks}. NH can cause a range of symptoms like sleep disturbances, morning headaches, chronic fatigue, mood changes and is associated with cardiac arrhythmias, which can result in "death-in-bed syndrome" \cite{graveling2017risks}. In healthy subjects, hypoglycemia triggers awakening, however, diabetes patients are often unable to wake up when their blood glucose drops, showing the need for reliable prediction methods \cite{schultes2007defective}. 

A prominent approach is the integration of continuous glucose monitoring (CGM) into decision-support systems to assist diabetes patients with their self-management \cite{vettoretti2018continuous}. Decision-support systems enable a prediction at bedtime about the risk of adverse events, such as NH, and recommend measures to reduce complications \cite{cappon2017wearable}. Within these systems, machine learning (ML) models using CGM data have been successfully developed in Type 1 diabetes (T1D) patients to predict NH \cite{bertachi2020prediction,parcerisas2022machine,berikov2022machine,guemes2019predicting}. Despite accurate results in this population, research for T2D remains scarce. To the best of our knowledge, only a single study \cite{kronborg2022bedtime} predicts the risk of NH for T2D patients with a population limited to insulin-treated patients and CGM-based data. 

While CGM data is the most promising approach in predicting NH for T1D \cite{mujahid2021machine,felizardo2021data,zhang2023data}, its implementation on a large scale for T2D is not feasible due to high costs and monitoring frequency, leading to a higher risk of undetected NH episodes in T2D \cite{oyaguez2021cost,wood2018continuous}. As a result, additional features are needed such as physical activity and carbohydrate intake that are related to T2D prevalence and the occurrence of NH \cite{woodward2009nocturnal,wilson2015factors}. The use of lifestyle features instead of CGM features for NH prediction could reduce healthcare costs and help healthcare providers to develop further interventions. However, there is no information about the effectiveness of lifestyle features for NH prediction since most studies have focused on CGM features or combinations of multiple feature categories.

The limited availability of CGM data for T2D patients also impedes the most conventional approach of building personalized predictive models \cite{bertachi2020prediction,berikov2022machine,guemes2019predicting,mosquera2020predicting} for NH prediction. Personalized models require sufficient training data, spanning multiple days of tracking for each patient, which is generally the case for T1D patients as they are equipped with CGM in their daily life. However, typical CGM trackers for T2D patients have a short durability of 14 days \cite{oyaguez2021cost}, resulting in insufficient individual data sets for building accurate personalized models. As a result, alternative approaches such as population models are needed. Population models offer a cost and time-efficient way to assess the risk of NH by creating a common system for all users and utilizing baseline patient characteristics \cite{parcerisas2022machine}. To determine if population-based models are a viable alternative to personalized models it is important to investigate whether population models can accurately predict NH events in T2D.

Recent reviews \cite{mujahid2021machine,felizardo2021data,zhang2023data} further showed that the performance of different ML approaches varies with the study population, outcome definition, and training and validation approaches \cite{zhang2023data}. Especially, when investigating a new population, comparing and improving the predictive performance of multiple ML models and assessing their applicability in given clinical situations remains an important challenge.

Therefore, this study aims to address the challenges faced in NH prediction for a new population of T2D patients and to fill the gap in the literature, which has primarily focused on T1D patients. Specifically, this study investigates the use of lifestyle features, such as physical activity and carbohydrate intake, to predict, at bedtime, the risk of NH in T2D patients as an alternative to CGM data.

First, data processing, feature extraction, and feature selection have been applied. Second, multiple ML algorithms have been trained and optimized for two subsets of features with one including (Full data condition) and one excluding CGM (Lifestyle condition) to build population models. Third, the predictive performance between ML algorithms and also between the two data conditions have been compared. Last, the results have been critically evaluated to assess their impact on the existing literature and their contribution to the diabetes management of T2D patients.

