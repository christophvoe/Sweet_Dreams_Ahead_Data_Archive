\section{Discussion}
\label{Discussion}

This study showed that ML-based prediction models can be successfully used to mitigate the risk of hypoglycemia in T2D patients. Furthermore, this study demonstrated that important contributions can be made toward hypoglycemia prevention in T2D by investigating models that excluded CGM data for model building.

This study contributes to the literature on predicting NH in T2D patients by incorporating a larger study population with more than double the sample size compared to a previous study. The study also includes a broader group of patients, including those without insulin treatment, and utilizes multiple data sources such as CGM, physical activity, dietary information, and patient characteristics. These findings enhance the generalizability of ML-based prediction models for T2D patients and further improve the development of preventative interventions.

To further assess the applicability in clinical situations various model-building decisions were tested in this study, which is specifically needed when dealing with unknown populations. The study included multiple ML algorithms, several tuning strategies, and distinct sampling approaches to compare and improve the predictive performance of the models. The results showed that the predictive performance varied across different ML algorithms, indicating the need to test multiple methods to see which approach fits best to the given data characteristics. For instance, the RF algorithm demonstrated superior performance in this study, indicating that it might be a suitable method for predicting NH in T2D patients. However, it is important to note that the performance of the ML algorithms depends on study characteristics, such as the study population, outcome definition, and validation approaches. Therefore, relying on a single ML algorithm may not be sufficient to achieve high predictive performance.

Additionally, in NH prediction, class imbalance is a common issue, and sampling approaches, such as SMOTE and Up-sampling, have been suggested to handle it. However, in this study, these techniques were found to have no advantageous impact on the model's performance, implying that other methods may be more effective. Specifically, cost-sensitive learning was found to significantly improve the model's predictive performance, indicating its potential as a reliable alternative. Moreover, suggesting in order to avoid costly misclassifications, especially in clinical settings, cost-sensitive learning should be considered an essential tool for NH prediction.

As the main research question, this study investigated the performance of ML models using only lifestyle features and patient characteristics, which is a novel approach in NH prediction. To investigate the difference between the two approaches, this study compared the predictive performances of models trained on lifestyle features and patient characteristics with those trained on the full data, including CGM. The results of this study suggested that the inclusion of CGM features consistently led to higher AUC values compared to the estimates in the Lifestyle conditions, while the AUC values for the Lifestyle condition still yielded good results compared to similar studies. In addition, the estimates for SENS of the Lifestyle conditions led to similar estimates compared to the Full data condition. This indicated that while the use of CGM data clearly improves predictive performance, lifestyle features can be a useful addition to NH prediction if CGM is not available. 

These results also highlight the potential of using relatively easy and inexpensive ways to assess important predictors for NH detection in T2D and also emphasize the need for further research to identify other cost-effective and easily accessible lifestyle features that can improve model performance. Based on the findings of this study effective intervention strategies can be developed. Exemplary, the results of this study offer the chance to include prediction models based on different data sources in decision support systems for T2D patients.

This study also aimed to investigate whether population-based models can provide comparable or even superior results to personalized models in predicting NH in T2D patients. The results demonstrated that the information from the population models can be used to accurately predict NH events in individuals with T2D, which is particularly important since personalized models may not be feasible for all T2D patients. However, the results showed some variability in their estimates due to the relatively small sample size. Therefore, this study emphasizes the need for large databases that include time-dependent data on patient lifestyle, CGM values, and baseline characteristics, which will help to get more accurate predictions.

Despite the promising results, there are some limitations that need to be addressed. 
One of the main drawbacks is associated with the scarce amount of available data, which greatly restricts the training and validation of the classifiers, resulting in high SD estimates (as presented in Table \ref{tab:results_all}). Large databases are crucial when building population models, requiring the accumulation of data from different populations using similar monitoring devices.
Another limitation is that the system's data quality is highly dependent on patients' commitment and can be prone to errors. For example, information such as medication dosing, carbohydrate intake, and consistent use of the wristband may not be accurately captured under free-living conditions, resulting in questionable data quality. 
Also, the dataset lacks additional lifestyle features, such as heart rate, and other surrogates of physical activity that could improve the predictions. The lifestyle features in this study, such as dietary entries, were found to be time-consuming for patients and of questionable quality. To improve the accuracy of the models and reduce the burden on patients, it is important to collect less time-consuming and less error-prone lifestyle features.
Finally, the algorithm was trained using a pre-defined bedtime of 12:00 PM, which may not be applicable to everyone. Different working patterns and sleep schedules should be included to increase the accuracy of the system.  As a solution, bedtime for each period could be detected individually with the use of trackers.

The results of this study provide several considerations for future research in the field of diabetes management. Firstly, future studies should investigate additional outcome variables, such as hyperglycemia events, and explore the use of shorter prediction horizons, which can help identify patients at risk of a wider range of complications.

Secondly, an in-silico study could be conducted to assess the efficacy of interventions such as recommending bedtime snacks with varying sizes based on different ML models or data conditions, as previously done for T1D patients \cite{mosquera2020predicting,parcerisas2022machine}.

Finally, this study could serve as a basis for further clinical studies, which evaluate the effectiveness of ML-based algorithms embedded in decision support systems under real-life conditions. For instance, a clinical trial could investigate how an ML algorithm integrated into a smartphone app recommends bedtime carbohydrates if NH is predicted. The study could include three groups: a control group receiving only monitoring, a group receiving recommendations based on only lifestyle features and patient characteristics, and a group receiving recommendations based on the full data including CGM.

These future studies have the potential to improve the accuracy and efficacy of diabetes management and create opportunities for more personalized and effective treatment options.
