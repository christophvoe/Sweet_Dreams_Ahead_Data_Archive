\section{Conclusion}
\label{Conclusion}

In conclusion, this study provides valuable insights into the use of ML-based prediction models for hypoglycemia prevention in a new population of T2D patients. In detail, this study suggests that the use of CGM data consistently improves the predictive performance of NH prediction models. However, incorporating lifestyle features and patient characteristics in population-based models can also provide useful information, especially in situations where CGM data is not available. The findings of this study have significant implications for healthcare providers in developing effective intervention strategies to optimize patient outcomes and highlight the need for further research to identify cost-effective and easily accessible lifestyle features that can improve prediction performance. 

\section{Abbreviations}
Continuous Glucose Monitoring; NH, Nocturnal hypoglycemia; T2D, Type 2 Diabetes; CGM, ; ML, Machine Learning; AUC, Area under Curve; SENS, Sensitivity; SPEC, Specificity; RF, Random Forest; T1D, Type 1 Diabetes; SVM, Support Vector Machine; ANN, Artificial Neural Network; XGB, eXtreme Gradient Boosting; DIALECT, Diabetes and Lifestyle Cohort Twente; LBGI, Low BLood GLucose Index; HBGI, High BLood GLucose Index; GRADE, Glycaemic Risk
Assessment Diabetes Equation; GMI, Glucose Management Indicator; COB, carbohydrate onboard model; AOB, activity onboard model; HbA1c, Hemoglobin A1c; BMI, Body Mass Index; MI, Mutual Information; Lasso, Lasso Logistic Regression; SMOTE, Synthetic Minority Over-sampling Technique; ROC, Receiver operating characteristic; SD, Standard Deviation.
\label{Abbreviations}